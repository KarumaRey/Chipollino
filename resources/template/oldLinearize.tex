\begin{frame}{Линеаризация}
  \begin{block}{\bf Определение}
    Если регулярное выражение $r\in\RegExp$ содержит $n$ вхождений букв алфавита $\Sigma$, тогда линеаризованное регулярное выражение $\Linearize(r)$ получается из $r$ приписыванием $i$-ой по счёту букве, входящей в $r$, индекса $i$.
  \end{block} % descriptive documentation

  \begin{exampleblock}{\bf Пример}
    Рассмотрим регулярное выражение:
    \[(\regexpstr{ba}\alter \regexpstr{b})\regexpstr{aa}(\regexpstr{a}\alter\regexpstr{ab})\star\] % the initial regexp placeholder displaystyle

    Его линеаризованная версия:
    \[(\regexpstr{b_{1}a_{2}}\alter \regexpstr{b_{3}})\regexpstr{a_{4}a_{5}}(\regexpstr{a_{6}}\alter\regexpstr{a_{7}b_{8}})\star\] % the linearised regexp placeholder displaystyle

  \end{exampleblock}

\end{frame}