% descriptive documentation :section
\section{Основные понятия}
\begin{frame}{Follow-эквивалентность}
  \begin{block}{\bf Определение}
    Пусть $R$ --- регулярное выражение. Положим $follow(a_{i}) = \{a_{j} | \exists w, u(wa_{i}a_{j}u \in\Lang(R))\}$.

    Follow-эквивалентность: состояния автомата Глушкова $a_{i}$ и $a_{j}$ follow-эквивалентны, если $follow(a_{i}) = follow(a_{j})$, и либо $a_{i}$, $a_{j}$ оба финальные, либо они оба не финальные.
  \end{block} % descriptive documentation
\end{frame}

\section{Follow-автомат (IlieYu)}
% descriptive documentation : frame
\begin{frame}{Конструкция автомата Илия-Ю (или follow-автомата)}
  \begin{block}{\bf Алгоритм построения $\IlieYu(r)$}
    \begin{itemize}
      \item Построить автомат Глушкова ($\Glushkov$);
      \item Объединить follow-эквивалентные состояния.
    \end{itemize}
  \end{block} % descriptive documentation
\end{frame}

\begin{frame}{Пример Follow-автомата (IlieYu)}\only<3> {\vspace{-5pt}}
  \vspace{-5pt}
  \only<1-3> {
    Исходное регулярное выражение:

    %template_initial_regex % the initial regexp placeholder displaystyle
  }
  \only<1> {
    Автомат Глушкова:

    %template_glushkov % the Glushkov diagram placeholder
  }
  \only<2> {
    Follow-отношения:
    %template_follow % the follow-relationships placeholder 1
  }
  \only<3>{
    Follow-автомат:

    %template_ilieyu % the IlieYu diagram placeholder
  }
\end{frame}

% overall documentation : section 
\section{Обсуждение}
\begin{frame}{Дополнительные сведения} \only<2> {\vspace{-5pt}}
  \only<1> {
    \begin{block}{\bf }
      \begin{itemize}
        \item Меньше позиционного автомата и, в среднем, быстрее вычисляется.
        \item Может быть вычислен за квдратичное время.
        \item Является частным от позиционного автомата.
      \end{itemize}
    \end{block}
    \begin{block}{\bf Связь с автоматом Томпсона}
      Follow-автомат (IlieYu) может быть получен из автомата Томпсона путем последовательного применения к нему следующих операций:
      \[\DeAnnote(\Minimize(\RemEps(\Annote(\Thompson(r)))))\] % the formula IlieYu placeholder displaystyle
    \end{block}
  }
  \only<2> {
    \begin{block}{\bf Теорема}
      Пусть $r$ -- взвешенное регулярное выражение над $K$. Если $K$ является $k$-замкнутым для автомата $\Thompson(r)$, то $\IlieYu(r)$ может быть вычислен за $O(mn)$ путем применения удаления $\empt$-переходов с последующей взвешенной минимизацией к  $\Thompson(r)$.
    \end{block}
    $K$ называется $k$-замкнутым, если $\forall a \in K$ , $\oplus_{n=0}^{k+1} a^n = \oplus_{n=0}^{k} a^n$. В более общем плане мы скажем, что $K$ замкнуто ($k$-замкнуто) для автомата $A$, если  аксиомы замкнутость ($k$-замкнутость) справедливы для всех весов цикла $A$.
    % я не уверена в правильности выражения "для всех весов цикла" (( вот источник: https://cs.nyu.edu/~mohri/pub/glush.pdf
  }
\end{frame}